\documentclass{article}
\usepackage{../fasy-hw}
\usepackage{ wasysym }

%% UPDATE these variables:
\renewcommand{\hwnum}{0}
\title{Discrete Structures, Homework 0}
\author{\todo{Payton Dupuis}}
\collab{n/a}
\date{due: 15 January 2021}

\begin{document}

\maketitle

This homework assignment should be
submitted as a single PDF file both to D2L and to Gradescope.

General homework expectations:
\begin{itemize}
    \item Homework should be typeset using LaTex.
    \item Answers should be in complete sentences and proofread.
    \item You will not plagiarize.
    \item List collaborators at the start of each question using the
        \texttt{collab} command.
\end{itemize}

% ============================================
% ============================================
\nextprob{Getting to Know You}
\collab{n/a}
% ============================================
% ============================================

Answer the following questions:
\begin{enumerate}
    \item What is your elevator pitch?  Describe yourself in 1-2
        sentences.
        \paragraph{Answer} \todo{Hi, my name is Payton and I am from Polson, Montana. I am a junior majoring in Chemical Engineering with a mior in Computer Science.}

    \item What was your favorite college class so far, and why?
        \paragraph{Answer} \todo{My favorite class so far is Molecular and Cellular Biology because it was the first biology class that I have taken and it is similar to what I want to go into.}

    \item What was your least favorite college class so far, and why?
        \paragraph{Answer} \todo{My least favorite class is Reactor Design as I will not be using reactors in my future job so it seems irrelevant}

    \item Why are you interested in taking this course? (If your answer is
        `because I am required to by my major/minor', perhaps answer the
        alternative question: Why are you in your major?)
        \paragraph{Answer} \todo{Although this class is required, I am interested in this class to learn about the math logic behind programming.}

    \item What is your biggest academic or research goal for this semester (can
        be related to this course or not)?
        \paragraph{Answer} \todo{My biggest research goal is to get accepted to an REU in the biomedical sciences that will occur this summer.}

    \item What do you want to do after you graduate?
        \paragraph{Answer} \todo{After I graduate, I want to go to graduate school to obtain a degree in the Biomedical Sciences.}

    \item What was the most challenging aspect of blended or online courses?
        \paragraph{Answer} \todo{The most challenging part of online is that I tend to get intense migraines after being on the computer for three hours or more.}

    \item What do you like about blended or online courses?
        \paragraph{Answer} \todo{I like online because I can watch lecture videos at the speed that is best for me.}

\end{enumerate}

% ============================================
% ============================================
\nextprob{Administrative Tasks}
\collab{n/a}
% ============================================
% ============================================

Please do the following:
\begin{enumerate}
    \item Write this homework in LaTex. This will not be strictly enforced for
        this homework, but it is strongly encouraged.  Future homeworks will not
        be graded if they are not typeset in LaTex.
    \item Update your photo on D2L to be a recognizable headshot of you.
    \item Sign up for the class discussion board.
\end{enumerate}

\paragraph{Answer}


\todo{I have logged into the discord server, my D2L is a picture of myself, and I am currently writing in LaTex.}

% ============================================
% ============================================
\nextprob{Plagiarism}
\collab{n/a}
% ============================================
% ============================================


In this class, please properly cite all resources that you use.  To refresh your
memory on what plagiarism is, please complete the plagiarism tutorial found
here: \url{http://www.lib.usm.edu/plagiarism_tutorial}.  If you have observed
plagiarism or cheating in a classroom (either as an instructor or as a student),
explain the situation and how it made you feel.  If you have not experienced
plagiarism or cheating or if you would prefer not to reflect on a personal
experience, find a news article about plagiarism or cheating and explain how you
would feel if you were one of the people involved.

\paragraph{Answer}

\todo{During my calc 3 class final, I caught I student taking pictures of the final with his phone. This action increased my already high anxiety due to the test. I debated for about ten minutes how I should tell the professor what happened without disturbing the other students around me. I finally raised my hand and got the professor to come over. At that time, I told him that the student to the right of me took pictures of the exam. The professor then went to the student right after I told him and confronted the student. I did not feel good after that as the student clearly knew that I was the one that told on them. The student went to the front of the room and left a few minutes later. I managed to finish the exam but I did not feel safe when I was walking home as it was very dark outside. I did get home safely and have not seen that student since.}

% ============================================
% ============================================
\nextprob{Exams}
\collab{n/a}
% ============================================
% ============================================

I am exploring various options for exams for this semester: take-home,
in-person, synchronous online.  If you have any comments about what worked or
did not work in previous semesters with respect to classes in blended and online
settings, please share that here.

\paragraph{Answer}


\todo{I do not have a preference of which form the exam. I do find that take-home or in-person would be easier.}




% ============================================
% ============================================
\nextprob{Terminology}
\collab{\todo{}}
% ============================================
% ============================================

Sometimes concepts are taught more than once throughout the curriculum.  Each
time you encounter a concept, your understanding of it is deepened.
For each of the terms or statements below, describe in your own words what they
mean.  This will not be graded for correctness, just whether you have done it or
not.  Answering these to the best of your ability will help the instructor and
TA understand the base knowledge of the students in this class.
I encourage you to meet with a partner or two to refresh yourself on what these
terms mean (if you do, be sure to update the \texttt{collab} command
above!).  However, please keep the web searches to a minimum for this one!  It
is acceptable to answer `I have not heard of this term' or `I have heard of
this, but do not remember what it means.'
\begin{enumerate}
    \item $f(n)$ is $O(n^2)$.
        \paragraph{Answer}
        \todo{I am not sure if I have seen this before. The "is" is confusing me as I am not sure if it is = or something else.}
    \item $f(n)$ is $O(g(n))$.
        \paragraph{Answer}
        \todo{Again, I am not sure if "is" is an equal sign or something else.}
    \item $f(n)$ is $\Omega(n^3)$.
        \paragraph{Answer}
        \todo{Again, I am not sure if "is" is an equal sign or something else.}
    \item $f(n)$ is $\Theta(n\log n)$.
        \paragraph{Answer}
        \todo{Again, I am not sure if "is" is an equal sign or something else.}
    \item Binomial Coefficients
        \paragraph{Answer}
        \todo{I have heard of this before but can not recall the definition.}
    \item Four Color Theorem
        \paragraph{Answer}
        \todo{I have not heard of this before.}
    \item Graph
        \paragraph{Answer}
        \todo{A graph is a picture representation of the data.}
    \item Modus Ponens
        \paragraph{Answer}
        \todo{I have not heard of this before.}
    \item Proof by Counter-example
        \paragraph{Answer}
        \todo{I have not heard of this before.}
    \item Proof by Example
        \paragraph{Answer}
        \todo{I have heard of this before but can not recall the definition.}
    \item Proof by Induction
        \paragraph{Answer}
        \todo{I have heard of this before but can not recall the definition.}
    \item Recurrence Relation
        \paragraph{Answer}
        \todo{I have not heard of this before.}
    \item Recursive Algorithm
        \paragraph{Answer}
        \todo{I have not heard of this before.}
    \item Searching Algorithms
        \paragraph{Answer}
        \todo{I have not heard of this before.}
    \item Sorting Algorithms
        \paragraph{Answer}
        \todo{I have not heard of this before.}
    \item Tree
        \paragraph{Answer}
        \todo{I have heard of this before but do not know if it is what I am thinking about.}
\end{enumerate}

% ============================================
% ============================================
\nextprob{Real Numbers}
\collab{\todo{}}
% ============================================
% ============================================

Review the Properties of Real Numbers in Appendix A.  If any are unfamiliar or
confusing, please post a question in the group discussion board.  In the
write-up, write the following: `I have reviewed all properties of real numbers
in Appendix A.`

\paragraph{Answer}

\todo{I have reviewed all properties of real numbers in Appendix A.}


% ============================================
% ============================================
\nextprob{Georg Cantor}
\collab{\todo{}}
% ============================================
% ============================================

Write a short (1-2 paragraph) biography of Georg Cantor.
\textbf{In your own words}, describe who they are and why they are important in
the history of computer science.  If you use external resources, please provide
proper citations.

\paragraph{Answer}

% ============================================

\todo{Georg Cantor was born in Russia in 1845. Cantor found his love for math and eventually overrulled his father's want for him to become an engineer. While attending the University of Berlin, Cantor's mathematic professors would all have a great impact on his life. After obtaining a doctorate degree he got a teaching position at the University of Halle where he would stay. During his time at UH, Cantor wrote papers on the theory of numbers and went on to the theory of trigonometric series. Georg's discovery of the set theory which is important to computer science as the theory is "a source of fundamental ideas." During Georg's later years, he would help create the first international congress of mathematics.    Sources: https://www.britannica.com/biography/Georg-Ferdinand-Ludwig-Philipp-Cantor and "Set Theory for Computer Science" by Glynn Winskel}

% ============================================

\end{document}

